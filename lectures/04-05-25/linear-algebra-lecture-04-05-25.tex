\documentclass[11pt]{article}
\usepackage{amsmath, amssymb, amsthm}
\usepackage[margin=1in]{geometry}
\usepackage{tcolorbox}

% Theorem Environments
\newtheorem{theorem}{Theorem}[section]
\newtheorem{lemma}[theorem]{Lemma}
\newtheorem{corollary}[theorem]{Corollary}
\newtheorem{proposition}[theorem]{Proposition}
\newtheorem{conjecture}[theorem]{Conjecture}

\theoremstyle{definition}
\newtheorem{definition}[theorem]{Definition}
\newtheorem{example}[theorem]{Example}
\newtheorem{exercise}[theorem]{Exercise}

\theoremstyle{remark}
\newtheorem{remark}[theorem]{Remark}

% Custom Commands
\newcommand{\F}{\mathbb{F}}
\newcommand{\R}{\mathbb{R}}
\newcommand{\N}{\mathbb{N}}
\newcommand{\C}{\mathbb{C}}
\newcommand{\Z}{\mathbb{Z}}
\newcommand{\Q}{\mathbb{Q}}
\newcommand{\M}[2]{\mathrm{M}_{#1 \times #2}} % Matrices m x n
\newcommand{\Mn}[1]{\mathrm{M}_{#1}} % Square matrices n x n
\newcommand{\Poly}[1]{\mathcal{P}_{#1}} % Polynomials up to degree n
\newcommand{\Rn}[1]{\R^{#1}}
\newcommand{\Fn}[1]{\F^{#1}}
\newcommand{\coord}[2]{[#1]_{#2}} % Coordinate vector [v]_S
\newcommand{\img}{\mathrm{Im}} % Image of a map
\newcommand{\nul}{\mathrm{Null}} % Null space
\newcommand{\col}{\mathrm{Col}} % Column space
\newcommand{\rank}{\mathrm{rank}}
\newcommand{\id}{\mathrm{Id}}
\newcommand{\tr}{^\mathsf{T}} % Transpose

% Administrative Note Box
\newtcolorbox{adminnote}[1][]{%
  colback=blue!5!white, colframe=blue!50!black, arc=2mm, boxrule=0.5pt,
  title=Administrative Note, fonttitle=\bfseries\sffamily,
  #1
}

\title{Linear Algebra: Isomorphisms, Change of Basis, and Linear Mappings}
\author{Lecture Notes} % Placeholder
\date{\today} % Placeholder

\begin{document}
\maketitle

\begin{adminnote}
    Welcome everyone! A couple of quick notes before we dive into today's material:
    \begin{itemize}
        \item Please remember the homework exercise related to proving that the composition of isomorphisms (specifically $\Psi_V^{-1} \circ \Phi_U$ from our discussion on equivalent spaces) is indeed an isomorphism. You'll need to verify linearity, injectivity, and surjectivity.
        \item Later today, we'll encounter an integral operator $L(f)(x) = \int_a^x f(s) ds$. Showing this operator is injective (one-to-one) is a nice exercise that connects linear algebra concepts back to calculus – this will also be part of your homework.
        \item We took a short break during the original lecture recording after discussing the proof of the coordinate change formula.
    \end{itemize}
\end{adminnote}

\section{Recap: Coordinate Mappings and Isomorphism}

We've been exploring vector spaces, which are fundamental structures in mathematics. While the abstract definitions give us power and generality, it's often incredibly useful to have a concrete way to represent and compute with vectors. This is where the idea of coordinates comes in.

\subsection{Coordinate Vectors}

Imagine a finite-dimensional vector space $V$ over a field $\F$. If we choose an **ordered basis** $S = \{u_1, u_2, \dots, u_n\}$ for $V$, then any vector $v \in V$ can be written as a unique linear combination of these basis vectors:
\[ v = c_1 u_1 + c_2 u_2 + \dots + c_n u_n \]
The coefficients $(c_1, c_2, \dots, c_n)$ are the **coordinates** of $v$ relative to the basis $S$. We collect these coordinates into a column vector:
\begin{definition}[Coordinate Vector]
    Let $V$ be a vector space over $\F$ with an ordered basis $S = \{u_1, \dots, u_n\}$. For any $v \in V$, if $v = c_1 u_1 + \dots + c_n u_n$, the **coordinate vector of $v$ relative to $S$** is
    \[ \coord{v}{S} = \begin{pmatrix} c_1 \\ c_2 \\ \vdots \\ c_n \end{pmatrix} \in \Fn{n}. \]
\end{definition}

\begin{remark}
    The choice of basis $S$ is crucial! Changing the basis will change the coordinate vector for the same abstract vector $v$. The subscript $S$ emphasizes this dependency.
\end{remark}

\subsection{The Coordinate Map as an Isomorphism}

The process of finding coordinates defines a mapping from the abstract vector space $V$ to the familiar concrete space $\Fn{n}$:
\[ \Phi_S: V \to \Fn{n} \quad \text{defined by} \quad \Phi_S(v) = \coord{v}{S}. \]
This mapping is incredibly special. It doesn't just pair vectors; it preserves the entire vector space structure.

\begin{theorem}[Coordinate Map Isomorphism]
    Let $V$ be an $n$-dimensional vector space over $\F$ with ordered basis $S$. The coordinate map $\Phi_S: V \to \Fn{n}$ is an **isomorphism**.
\end{theorem}

Recall that an isomorphism is a linear map that is both one-to-one (injective) and onto (surjective).
\begin{itemize}
    \item \textbf{Linearity:} $\Phi_S(av + bw) = a\Phi_S(v) + b\Phi_S(w)$. This follows from the properties of vector addition and scalar multiplication combined with the uniqueness of coordinate representations.
    \item \textbf{One-to-one:} If $\Phi_S(v) = \Phi_S(w)$, then $\coord{v}{S} = \coord{w}{S}$, which implies $v=w$ due to the uniqueness of the linear combination for a given basis. Equivalently, the kernel is just $\{0_V\}$.
    \item \textbf{Onto:} For any coordinate vector $(c_1, \dots, c_n)\tr \in \Fn{n}$, the vector $v = c_1 u_1 + \dots + c_n u_n$ exists in $V$ and maps to it, i.e., $\Phi_S(v) = (c_1, \dots, c_n)\tr$.
\end{itemize}
This isomorphism tells us that any $n$-dimensional vector space $V$ over $\F$, no matter how abstract it seems (like spaces of polynomials or functions), behaves structurally exactly like $\Fn{n}$.

\section{Isomorphism Between Vector Spaces}

The idea that different-looking spaces can have the same underlying structure is powerful. Isomorphism formalizes this.

\begin{theorem}[Equivalence via Dimension]
    Let $U$ and $V$ be finite-dimensional vector spaces over the same field $\F$. Then $U$ is isomorphic to $V$ (denoted $U \cong V$) if and only if $\dim(U) = \dim(V)$.
\end{theorem}

\begin{proof}[Proof Sketch]
    ($\Rightarrow$) If $U \cong V$, there exists an isomorphism $T: U \to V$. Isomorphisms preserve linear independence and spanning sets, so they map bases to bases. Thus, $\dim(U) = \dim(V)$.

    ($\Leftarrow$) Suppose $\dim(U) = \dim(V) = n$. Let $S_U$ be a basis for $U$ and $S_V$ be a basis for $V$. We have coordinate isomorphisms $\Phi_{S_U}: U \to \Fn{n}$ and $\Phi_{S_V}: V \to \Fn{n}$. Since $\Phi_{S_V}$ is an isomorphism, its inverse $\Phi_{S_V}^{-1}: \Fn{n} \to V$ is also an isomorphism.
    Consider the composition:
    \[ \Phi = \Phi_{S_V}^{-1} \circ \Phi_{S_U} : U \to V \]
    This map takes a vector $u \in U$, finds its coordinates $\coord{u}{S_U}$ in $\Fn{n}$, and then uses $\Phi_{S_V}^{-1}$ to construct the vector $v \in V$ that has those same coordinates relative to the basis $S_V$.
    Since $\Phi_{S_U}$ and $\Phi_{S_V}^{-1}$ are both isomorphisms (linear, 1-1, onto), their composition $\Phi$ is also an isomorphism. (Verifying this is a useful exercise, as mentioned in the administrative notes).
\end{proof}

\subsection{Why Isomorphism Matters: Solving Problems}

This theorem is more than just a theoretical curiosity. It provides a practical strategy:
\begin{enumerate}
    \item Have a problem in an abstract $n$-dimensional space $V$?
    \item Choose a basis $S$ for $V$.
    \item Use the coordinate map $\Phi_S$ to translate the problem into $\Fn{n}$. This often involves converting vectors into coordinate vectors.
    \item Solve the problem in $\Fn{n}$ using familiar tools (like matrices, row reduction, etc.).
    \item Use the inverse map $\Phi_S^{-1}$ to translate the solution back to the original space $V$.
\end{enumerate}
Because isomorphisms preserve structure (like linear independence), the solution translated back will be valid in $V$.

Let's see this in action.

\begin{example}[Checking Linear Independence via Coordinates (Original Example 1)]
    Consider the vector space $\Poly{2}(\R)$, the space of real polynomials of degree at most 2. Its dimension is 3. Let's determine if the set $T = \{p_1(x), p_2(x), p_3(x)\}$ is linearly independent, where:
    \[ p_1(x) = 1+x, \quad p_2(x) = x+x^2, \quad p_3(x) = 1+x^2 \]
    We could try to solve $c_1 p_1(x) + c_2 p_2(x) + c_3 p_3(x) = 0$ directly, but let's use the isomorphism strategy.

    \textbf{Step 1 \& 2: Choose Basis and Translate.}
    The standard basis for $\Poly{2}(\R)$ is $S = \{1, x, x^2\}$. This is convenient because finding coordinates is easy. Let $\Phi_S: \Poly{2}(\R) \to \Rn{3}$ be the coordinate map. We find the coordinate vectors for our polynomials:
    \begin{align*} %\label{eq:polycoords} % Label inside align* has no effect
        p_1(x) = 1 \cdot 1 + 1 \cdot x + 0 \cdot x^2 &\implies \coord{p_1}{S} = \begin{pmatrix} 1 \\ 1 \\ 0 \end{pmatrix} \\
        p_2(x) = 0 \cdot 1 + 1 \cdot x + 1 \cdot x^2 &\implies \coord{p_2}{S} = \begin{pmatrix} 0 \\ 1 \\ 1 \end{pmatrix} \\
        p_3(x) = 1 \cdot 1 + 0 \cdot x + 1 \cdot x^2 &\implies \coord{p_3}{S} = \begin{pmatrix} 1 \\ 0 \\ 1 \end{pmatrix}
    \end{align*}
    The question now becomes: Is the set $\{\coord{p_1}{S}, \coord{p_2}{S}, \coord{p_3}{S}\}$ linearly independent in $\Rn{3}$?

    \textbf{Step 3: Solve in $\Rn{3}$.}
    We check the independence of these column vectors by forming a matrix and row reducing (or checking the determinant):
    \[ A = \begin{pmatrix} 1 & 0 & 1 \\ 1 & 1 & 0 \\ 0 & 1 & 1 \end{pmatrix} \]
    Let's row reduce:
    \[ \begin{pmatrix} 1 & 0 & 1 \\ 1 & 1 & 0 \\ 0 & 1 & 1 \end{pmatrix} \xrightarrow{R_2 \leftarrow R_2 - R_1} \begin{pmatrix} 1 & 0 & 1 \\ 0 & 1 & -1 \\ 0 & 1 & 1 \end{pmatrix} \xrightarrow{R_3 \leftarrow R_3 - R_2} \begin{pmatrix} 1 & 0 & 1 \\ 0 & 1 & -1 \\ 0 & 0 & 2 \end{pmatrix} \]
    Since we have 3 pivots (or the determinant is $1(1\cdot 1 - (-1)\cdot 1) - 0 + 1(1\cdot 1 - 1\cdot 0) = 2 \neq 0$), the columns are linearly independent.

    \textbf{Step 4: Translate Back.}
    Because the coordinate vectors $\{\coord{p_1}{S}, \coord{p_2}{S}, \coord{p_3}{S}\}$ are linearly independent in $\Rn{3}$, and $\Phi_S$ is an isomorphism (preserving linear independence), the original set of polynomials $T = \{1+x, x+x^2, 1+x^2\}$ must be linearly independent in $\Poly{2}(\R)$.

    \textit{Why basis $S=\{1,x,x^2\}$?} It was chosen purely for convenience. Finding the coordinates (the coefficients $c_1, c_2, c_3$) is trivial by inspection for this basis. Any other basis would work, but the coordinate finding step might be more involved.
\end{example}

\section{Change of Basis}

We've seen that coordinates depend on the chosen basis. What happens if we switch from one basis to another? How are the coordinates related? This is the "change of basis" problem.

Let $V$ be an $n$-dimensional vector space with two ordered bases:
\begin{itemize}
    \item $S = \{u_1, u_2, \dots, u_n\}$ (the "old" basis)
    \item $T = \{v_1, v_2, \dots, v_n\}$ (the "new" basis)
\end{itemize}
We want to find a way to convert coordinates relative to $T$ into coordinates relative to $S$, and vice versa.

\begin{definition}[Change of Basis Matrix]
    The **change of basis matrix from $T$ to $S$** (or the transition matrix from $T$-basis vectors to $S$-coordinates) is the $n \times n$ matrix $P = P_{S \leftarrow T}$ whose columns are the coordinate vectors of the *new* basis vectors ($v_j$) expressed in terms of the *old* basis ($S$):
    \[ P = P_{S \leftarrow T} = \begin{pmatrix} | & | & & | \\ \coord{v_1}{S} & \coord{v_2}{S} & \dots & \coord{v_n}{S} \\ | & | & & | \end{pmatrix} \]
\end{definition}

\begin{remark}[Naming Convention]
    The notation $P_{S \leftarrow T}$ visually suggests converting from $T$-coordinates to $S$-coordinates (via multiplication: $\coord{x}{S} = P_{S \leftarrow T} \coord{x}{T}$). Some texts use $P_{T \to S}$ for the same matrix, focusing on the basis transformation itself (how $T$ is built from $S$). We'll primarily use the coordinate view: $P_{S \leftarrow T}$ converts $T$-coordinates to $S$-coordinates. The definition itself involves expressing the $T$-basis vectors ($v_j$) in the $S$-basis.
\end{remark}

Why is this matrix useful? Let $x \in V$. We can express $x$ using either basis:
\[ x = \sum_{j=1}^n c_j v_j \implies \coord{x}{T} = (c_1, \dots, c_n)\tr \]
\[ x = \sum_{i=1}^n d_i u_i \implies \coord{x}{S} = (d_1, \dots, d_n)\tr \]
We also know how to express the new basis vectors $v_j$ in the old basis $S$:
\[ v_j = \sum_{i=1}^n p_{ij} u_i \]
where $(p_{1j}, \dots, p_{nj})\tr = \coord{v_j}{S}$ is the $j$-th column of $P = P_{S \leftarrow T}$.
Substituting the expression for $v_j$ into the expression for $x$:
\[ x = \sum_{j=1}^n c_j v_j = \sum_{j=1}^n c_j \left( \sum_{i=1}^n p_{ij} u_i \right) = \sum_{i=1}^n \left( \sum_{j=1}^n p_{ij} c_j \right) u_i \]
Comparing this to $x = \sum_{i=1}^n d_i u_i$, by the uniqueness of coordinates in basis $S$, we must have:
\[ d_i = \sum_{j=1}^n p_{ij} c_j \]
This is precisely the formula for matrix multiplication! It says that the $i$-th coordinate in the $S$ basis ($d_i$) is the $i$-th entry of the product of the matrix $P$ and the coordinate vector in the $T$ basis ($(c_1, \dots, c_n)\tr$).

\begin{theorem}[Change of Coordinates Formula]
    Let $S$ and $T$ be two ordered bases for a finite-dimensional vector space $V$, and let $P = P_{S \leftarrow T}$ be the change of basis matrix from $T$ to $S$. Then for any vector $x \in V$:
    \[ \coord{x}{S} = P \coord{x}{T} \]
\end{theorem}

\begin{proposition}
    The change of basis matrix $P = P_{S \leftarrow T}$ is always invertible.
\end{proposition}
\begin{proof}
    The columns of $P$ are $\coord{v_1}{S}, \dots, \coord{v_n}{S}$. Since $T = \{v_1, \dots, v_n\}$ is a basis, it is a linearly independent set. The coordinate map $\Phi_S: V \to \Fn{n}$ is an isomorphism and thus preserves linear independence. Therefore, the set of coordinate vectors $\{\coord{v_1}{S}, \dots, \coord{v_n}{S}\}$ is linearly independent in $\Fn{n}$. An $n \times n$ matrix whose columns are linearly independent is invertible.
\end{proof}

Since $P$ is invertible, we can also write the transformation the other way:
\[ \coord{x}{T} = P^{-1} \coord{x}{S} \]
This gives rise to alternative terminology:
\begin{itemize}
    \item $P = P_{S \leftarrow T}$ is the **change of basis matrix** (its columns relate the bases, converting $T$-coords to $S$-coords).
    \item $P^{-1} = P_{T \leftarrow S}$ is the **change of coordinates matrix** (it transforms $S$-coords to $T$-coords).
\end{itemize}
Note that $P_{T \leftarrow S} = (P_{S \leftarrow T})^{-1}$.

\begin{example}[Finding a Change of Basis Matrix (Original Example 2)]
    Let $V = \Rn{2}$. Consider the bases:
    \begin{itemize}
        \item $S = \{u_1, u_2\} = \left\{ \begin{pmatrix} 2 \\ 3 \end{pmatrix}, \begin{pmatrix} -1 \\ 1 \end{pmatrix} \right\}$ (Old basis)
        \item $T = \{v_1, v_2\} = \left\{ \begin{pmatrix} 1 \\ 4 \end{pmatrix}, \begin{pmatrix} 3 \\ 2 \end{pmatrix} \right\}$ (New basis)
    \end{itemize}
    We want to find the change of basis matrix $P = P_{S \leftarrow T}$. Its columns are $\coord{v_1}{S}$ and $\coord{v_2}{S}$.

    \textbf{Column 1: Find $\coord{v_1}{S}$.} We need to find $c_1, c_2$ such that $v_1 = c_1 u_1 + c_2 u_2$:
    \[ \begin{pmatrix} 1 \\ 4 \end{pmatrix} = c_1 \begin{pmatrix} 2 \\ 3 \end{pmatrix} + c_2 \begin{pmatrix} -1 \\ 1 \end{pmatrix} = \begin{pmatrix} 2c_1 - c_2 \\ 3c_1 + c_2 \end{pmatrix} \]
    This gives the system:
    \begin{align*} 2c_1 - c_2 &= 1 \\ 3c_1 + c_2 &= 4 \end{align*}
    Adding the equations gives $5c_1 = 5 \implies c_1 = 1$. Substituting back gives $2(1) - c_2 = 1 \implies c_2 = 1$.
    So, $\coord{v_1}{S} = \begin{pmatrix} 1 \\ 1 \end{pmatrix}$.

    \textbf{Column 2: Find $\coord{v_2}{S}$.} We need to find $d_1, d_2$ such that $v_2 = d_1 u_1 + d_2 u_2$:
    \[ \begin{pmatrix} 3 \\ 2 \end{pmatrix} = d_1 \begin{pmatrix} 2 \\ 3 \end{pmatrix} + d_2 \begin{pmatrix} -1 \\ 1 \end{pmatrix} = \begin{pmatrix} 2d_1 - d_2 \\ 3d_1 + d_2 \end{pmatrix} \]
    This gives the system:
    \begin{align*} 2d_1 - d_2 &= 3 \\ 3d_1 + d_2 &= 2 \end{align*}
    Adding the equations gives $5d_1 = 5 \implies d_1 = 1$. Substituting back gives $2(1) - d_2 = 3 \implies d_2 = -1$.
    So, $\coord{v_2}{S} = \begin{pmatrix} 1 \\ -1 \end{pmatrix}$.

    \textbf{Result:} The change of basis matrix is:
    \[ P = P_{S \leftarrow T} = \begin{pmatrix} \coord{v_1}{S} & \coord{v_2}{S} \end{pmatrix} = \begin{pmatrix} 1 & 1 \\ 1 & -1 \end{pmatrix} \]
    Therefore, for any vector $x \in \Rn{2}$, its coordinates in the $S$ and $T$ bases are related by:
    \[ \coord{x}{S} = \begin{pmatrix} 1 & 1 \\ 1 & -1 \end{pmatrix} \coord{x}{T} \]
    And conversely:
    \[ \coord{x}{T} = P^{-1} \coord{x}{S} = \begin{pmatrix} 1 & 1 \\ 1 & -1 \end{pmatrix}^{-1} \coord{x}{S} = \frac{1}{-2} \begin{pmatrix} -1 & -1 \\ -1 & 1 \end{pmatrix} \coord{x}{S} = \begin{pmatrix} 1/2 & 1/2 \\ 1/2 & -1/2 \end{pmatrix} \coord{x}{S} \]
\end{example}

\begin{remark}[Proof of Coordinate Change Formula]
    The proof presented earlier by substituting $v_j = \sum_i p_{ij} u_i$ into $x = \sum_j c_j v_j$ is the standard way. The original lecture involved manipulating summations and identifying the result as $P \coord{x}{T}$. While correct, it can be notationally dense. The core idea is that the matrix $P$ encodes how to build the new basis vectors from the old ones, and this relationship directly translates coordinate representations.
\end{remark}

\section{A Survey of Linear Mappings}

We've focused heavily on isomorphisms, especially coordinate maps. But linear mappings are much broader. A linear mapping (or linear transformation) is simply a function between vector spaces that preserves the vector space operations.

\begin{definition}[Linear Mapping]
    Let $U$ and $V$ be vector spaces over the same field $\F$. A function $T: U \to V$ is a **linear mapping** if for all $u_1, u_2 \in U$ and all scalars $c \in \F$:
    \begin{enumerate}
        \item $T(u_1 + u_2) = T(u_1) + T(u_2)$ (Additivity)
        \item $T(c u_1) = c T(u_1)$ (Homogeneity)
    \end{enumerate}
    These two conditions can be combined into one: $T(c u_1 + d u_2) = c T(u_1) + d T(u_2)$ for all $u_1, u_2 \in U$ and $c, d \in \F$.
\end{definition}

Before diving into examples, let's review some general function concepts and introduce two specific ones relevant here.

\subsection{General Function Concepts}
Let $f: X \to Y$ be a function.
\begin{itemize}
    \item \textbf{Domain:} The set $X$.
    \item \textbf{Codomain:} The set $Y$.
    \item \textbf{Image (or Range):} $\img(f) = \{f(x) \mid x \in X\} \subseteq Y$.
    \item \textbf{One-to-one (Injective):} If $f(x_1) = f(x_2)$ implies $x_1 = x_2$. Distinct inputs map to distinct outputs. For linear maps $T: U \to V$, this is equivalent to $\ker(T) = \{0_U\}$.
    \item \textbf{Onto (Surjective):} If $\img(f) = Y$. Every element in the codomain is the image of at least one element in the domain.
\end{itemize}

\begin{definition}[Inverse Image]
    Let $f: X \to Y$ be a function and let $B \subseteq Y$. The **inverse image** of $B$ under $f$ is the set of all elements in the domain that map into $B$:
    \[ f^{-1}(B) = \{x \in X \mid f(x) \in B\} \subseteq X \]
\end{definition}
\begin{remark}
    This notation $f^{-1}(B)$ is defined even if $f$ is not invertible (i.e., not one-to-one and onto). It represents a set in the domain, not the application of an inverse function (which might not exist).
\end{remark}

\begin{definition}[Restriction]
    Let $f: X \to Y$ be a function and let $S \subseteq X$. The **restriction** of $f$ to $S$, denoted $f|_S$, is the function $f|_S: S \to Y$ defined by $f|_S(x) = f(x)$ for all $x \in S$. It's the same rule as $f$, but applied only to elements within the subset $S$.
\end{definition}

\subsection{Examples of Linear Mappings}

Let's explore the variety of linear mappings.

\begin{example}[Zero Mapping (Original Ex 1)]
    Let $U, V$ be vector spaces over $\F$. The **zero mapping** $Z: U \to V$ is defined by $Z(u) = 0_V$ for all $u \in U$.
    \begin{itemize}
        \item \textbf{Linear:} $Z(c u_1 + d u_2) = 0_V$ and $c Z(u_1) + d Z(u_2) = c 0_V + d 0_V = 0_V$. Yes.
        \item \textbf{One-to-one?} No, unless $U = \{0_U\}$. If $U$ contains non-zero vectors, they all map to $0_V$.
        \item \textbf{Onto?} No, unless $V = \{0_V\}$. The image is just $\{0_V\}$.
    \end{itemize}
\end{example}

\begin{example}[Identity Mapping (Original Ex 2)]
    Let $V$ be a vector space. The **identity mapping** $\id: V \to V$ is defined by $\id(v) = v$ for all $v \in V$.
    \begin{itemize}
        \item \textbf{Linear:} $\id(c v_1 + d v_2) = c v_1 + d v_2 = c \id(v_1) + d \id(v_2)$. Yes.
        \item \textbf{One-to-one?} Yes. $\ker(\id) = \{v \mid \id(v) = 0_V\} = \{0_V\}$.
        \item \textbf{Onto?} Yes. For any $v$ in the codomain, $v$ is the image of $v$ from the domain.
        \item \textbf{Operator:} A linear map from a space to itself ($T: V \to V$) is often called a linear **operator**.
    \end{itemize}
\end{example}

\begin{example}[Matrix Transformation (Original Ex 3)]
    Let $A \in \M{n}{m}(\F)$. The map $T_A: \Fn{m} \to \Fn{n}$ defined by $T_A(u) = Au$ (matrix multiplication) is linear.
    \begin{itemize}
        \item \textbf{Linear:} $T_A(c u_1 + d u_2) = A(c u_1 + d u_2) = c(Au_1) + d(Au_2) = c T_A(u_1) + d T_A(u_2)$. Yes, by properties of matrix multiplication.
        \item \textbf{One-to-one?} Yes, if and only if $\ker(T_A) = \nul(A) = \{0_m\}$. This happens iff the columns of $A$ are linearly independent (iff $\rank(A) = m$).
        \item \textbf{Onto?} Yes, if and only if $\img(T_A) = \col(A) = \Fn{n}$. This happens iff the columns of $A$ span $\Fn{n}$ (iff $\rank(A) = n$).
        \item \textbf{Sub-example:} $f: \Rn{2} \to \R$ by $f(x_1, x_2) = x_1 + x_2$. This is $T_A$ with $A = \begin{pmatrix} 1 & 1 \end{pmatrix}$.
            Here $m=2, n=1$. Rank is 1.
            Since $\rank(A) = 1 < m=2$, it's not one-to-one. E.g., $f(1,-1)=0, f(-1,1)=0$.
            Since $\rank(A) = 1 = n=1$, it is onto.
    \end{itemize}
\end{example}

\begin{example}[Matrix Multiplication Map (Original Ex 4)]
    Let $A \in \M{p}{m}(\F)$ and $B \in \M{n}{q}(\F)$ be fixed matrices. Define $T: \M{m}{n}(\F) \to \M{p}{q}(\F)$ by $T(X) = AXB$.
    \begin{itemize}
        \item \textbf{Linear:} $T(cX + dY) = A(cX+dY)B = A(cX)B + A(dY)B = c(AXB) + d(AYB) = cT(X) + dT(Y)$. Yes.
        \item \textbf{One-to-one? Onto?} Depends heavily on the properties of $A$ and $B$ (e.g., their ranks, invertibility if square). Analyzing this requires more advanced tools (like Kronecker products or vectorization).
    \end{itemize}
\end{example}

\begin{example}[Differentiation (Original Ex 5)]
    Let $C^1(\R)$ be the space of continuously differentiable functions $f: \R \to \R$, and $C(\R)$ be the space of continuous functions. The **differentiation map** $D: C^1(\R) \to C(\R)$ is defined by $D(f) = f'$ (the derivative of $f$).
    \begin{itemize}
        \item \textbf{Linear:} $D(cf + dg) = (cf+dg)' = cf' + dg' = cD(f) + dD(g)$. Yes, by standard derivative rules.
        \item \textbf{One-to-one?} No. $\ker(D) = \{f \in C^1(\R) \mid f' = 0\}$. These are precisely the constant functions. Since there are many non-zero constant functions, the kernel is non-trivial.
        \item \textbf{Onto?} Yes. For any continuous function $g \in C(\R)$, the Fundamental Theorem of Calculus (Part 1) guarantees that the function $f(x) = \int_0^x g(t) dt$ (plus any constant) is differentiable, belongs to $C^1(\R)$, and satisfies $f'(x) = g(x)$. Thus $D(f) = g$. Every continuous function has an antiderivative in $C^1(\R)$.
    \end{itemize}
\end{example}

\begin{example}[Integration Functional (Original Ex 6)]
    Let $C([a,b])$ be the space of continuous functions on $[a,b]$. Define $J: C([a,b]) \to \R$ by $J(f) = \int_a^b f(t) dt$.
    \begin{itemize}
        \item \textbf{Linear:} $J(cf+dg) = \int_a^b (cf(t)+dg(t)) dt = c\int_a^b f(t) dt + d\int_a^b g(t) dt = cJ(f) + dJ(g)$. Yes, by linearity of the definite integral.
        \item \textbf{Functional:} A linear map from a vector space to its scalar field ($\R$ or $\C$) is called a **linear functional**.
        \item \textbf{Onto?} Yes. For any $c \in \R$, consider the constant function $f(t) = c/(b-a)$ (assuming $b \neq a$). This $f$ is in $C([a,b])$ and $J(f) = \int_a^b \frac{c}{b-a} dt = \frac{c}{b-a} [t]_a^b = \frac{c}{b-a} (b-a) = c$.
        \item \textbf{One-to-one?} No. $\ker(J) = \{f \in C([a,b]) \mid \int_a^b f(t) dt = 0\}$. This space contains many non-zero functions (e.g., any function whose graph has equal signed area above and below the axis between $a$ and $b$, like $\sin(x)$ on $[0, 2\pi]$ if $a=0, b=2\pi$). The example sketched in the lecture involved constructing a piecewise linear function centered at $(a+b)/2$ with positive and negative parts cancelling out.
    \end{itemize}
\end{example}

\begin{example}[Integral Operator (Original Ex 7)]
    Define $L: C([a,b]) \to C([a,b])$ by $L(f)(x) = \int_a^x f(t) dt$. Note the output is a function of $x$.
    \begin{itemize}
        \item \textbf{Linear:} Similar justification to $J$ using linearity of integrals. Yes.
        \item \textbf{Onto?} No. The Fundamental Theorem of Calculus (Part 1) tells us that if $f$ is continuous, then $F(x) = L(f)(x) = \int_a^x f(t) dt$ is not just continuous, but also differentiable (specifically, $F'(x) = f(x)$). Furthermore, $F(a) = \int_a^a f(t) dt = 0$. Thus, the image $\img(L)$ is a subset of $C^1([a,b])$, consisting of continuously differentiable functions $F$ on $[a,b]$ such that $F(a)=0$. Since not all continuous functions are differentiable (e.g., $|x-a|$ if $a<b$), $L$ cannot be onto $C([a,b])$.
        \item \textbf{One-to-one?} Yes. Suppose $L(f) = 0$. This means the function $F(x) = \int_a^x f(t) dt$ is the zero function for all $x \in [a,b]$. Differentiating both sides with respect to $x$ (using FTC Part 1), we get $f(x) = F'(x) = 0$ for all $x \in (a,b)$. Since $f$ is continuous, $f(x)=0$ for all $x \in [a,b]$. Thus $\ker(L) = \{0\}$, and $L$ is one-to-one. (This was the homework exercise mentioned).
    \end{itemize}
\end{example}

\subsection{Examples of Non-Linear Mappings}

It's also important to recognize when a mapping is *not* linear. Linearity is quite a strict condition.

\begin{example}[Affine Map (Original Non-Linear Ex 1)]
    Let $A \in \M{n}{m}(\F)$ and $b \in \Fn{n}$ with $b \neq 0$. The map $f: \Fn{m} \to \Fn{n}$ defined by $f(u) = Au + b$ is called an **affine map**.
    \begin{itemize}
        \item \textbf{Not Linear:} A necessary condition for linearity is that $T(0_U) = 0_V$. Here, $f(0_m) = A(0_m) + b = 0_n + b = b$. Since $b \neq 0_n$, $f(0_m) \neq 0_n$, so the map cannot be linear. (It also fails additivity and homogeneity unless $b=0$).
    \end{itemize}
\end{example}

\begin{example}[Quadratic Matrix Map (Original Non-Linear Ex 2)]
    Let $V = \{A \in \Mn{n}(\R) \mid A\tr = A\}$ be the space of $n \times n$ real symmetric matrices. Define $f: V \to V$ by $f(A) = AA\tr$.
    \begin{itemize}
        \item \textbf{Well-defined target space?} If $A$ is symmetric, $A\tr = A$, so $f(A) = AA = A^2$. Is $A^2$ symmetric? $(A^2)\tr = (A\tr)\tr (A\tr) = A A\tr = A^2$ (using $(XY)\tr=Y\tr X\tr$ and $(X\tr)\tr = X$). More simply, $(A^2)\tr = (A\tr)^2$. If $A\tr=A$, then $(A^2)\tr = A^2$. Yes, the output is also symmetric. So the map $f(A)=AA\tr$ maps $V$ to $V$.
        \item \textbf{Not Linear:} Consider $f(A+B) = (A+B)(A+B)\tr = (A+B)(A\tr+B\tr) = AA\tr + AB\tr + BA\tr + BB\tr$. In general, this is not equal to $f(A)+f(B) = AA\tr + BB\tr$. For instance, if $A=I, B=I$, $f(A+B)=f(2I)=(2I)(2I)\tr=4I$. But $f(A)+f(B)=f(I)+f(I)=II\tr+II\tr=I+I=2I$. Since $4I \neq 2I$ (for $n \ge 1$), $f$ fails additivity. Also, $f(cA) = (cA)(cA)\tr = (cA)(c A\tr) = c^2 A A\tr = c^2 f(A)$, which is not $c f(A)$ unless $c=0, 1$. Thus, $f$ is not linear. (The original transcript simplified to $A^2$ assuming symmetric input, but the non-linearity holds regardless).
    \end{itemize}
\end{example}

These examples highlight that linearity involves preserving sums and scalar multiples exactly, without extra terms (like $+b$) or changes in powers (like $A^2$ or $c^2$).

\end{document}